% Options for packages loaded elsewhere
\PassOptionsToPackage{unicode}{hyperref}
\PassOptionsToPackage{hyphens}{url}
\PassOptionsToPackage{dvipsnames,svgnames,x11names}{xcolor}
%
\documentclass[
  letterpaper,
  DIV=11,
  numbers=noendperiod]{scrartcl}

\usepackage{amsmath,amssymb}
\usepackage{iftex}
\ifPDFTeX
  \usepackage[T1]{fontenc}
  \usepackage[utf8]{inputenc}
  \usepackage{textcomp} % provide euro and other symbols
\else % if luatex or xetex
  \usepackage{unicode-math}
  \defaultfontfeatures{Scale=MatchLowercase}
  \defaultfontfeatures[\rmfamily]{Ligatures=TeX,Scale=1}
\fi
\usepackage{lmodern}
\ifPDFTeX\else  
    % xetex/luatex font selection
  \setmainfont[]{Inter}
  \setsansfont[]{Inter}
  \setmathfont[]{Fira Math}
\fi
% Use upquote if available, for straight quotes in verbatim environments
\IfFileExists{upquote.sty}{\usepackage{upquote}}{}
\IfFileExists{microtype.sty}{% use microtype if available
  \usepackage[]{microtype}
  \UseMicrotypeSet[protrusion]{basicmath} % disable protrusion for tt fonts
}{}
\makeatletter
\@ifundefined{KOMAClassName}{% if non-KOMA class
  \IfFileExists{parskip.sty}{%
    \usepackage{parskip}
  }{% else
    \setlength{\parindent}{0pt}
    \setlength{\parskip}{6pt plus 2pt minus 1pt}}
}{% if KOMA class
  \KOMAoptions{parskip=half}}
\makeatother
\usepackage{xcolor}
\usepackage{soul}
\setlength{\emergencystretch}{3em} % prevent overfull lines
\setcounter{secnumdepth}{5}
% Make \paragraph and \subparagraph free-standing
\ifx\paragraph\undefined\else
  \let\oldparagraph\paragraph
  \renewcommand{\paragraph}[1]{\oldparagraph{#1}\mbox{}}
\fi
\ifx\subparagraph\undefined\else
  \let\oldsubparagraph\subparagraph
  \renewcommand{\subparagraph}[1]{\oldsubparagraph{#1}\mbox{}}
\fi


\providecommand{\tightlist}{%
  \setlength{\itemsep}{0pt}\setlength{\parskip}{0pt}}\usepackage{longtable,booktabs,array}
\usepackage{calc} % for calculating minipage widths
% Correct order of tables after \paragraph or \subparagraph
\usepackage{etoolbox}
\makeatletter
\patchcmd\longtable{\par}{\if@noskipsec\mbox{}\fi\par}{}{}
\makeatother
% Allow footnotes in longtable head/foot
\IfFileExists{footnotehyper.sty}{\usepackage{footnotehyper}}{\usepackage{footnote}}
\makesavenoteenv{longtable}
\usepackage{graphicx}
\makeatletter
\def\maxwidth{\ifdim\Gin@nat@width>\linewidth\linewidth\else\Gin@nat@width\fi}
\def\maxheight{\ifdim\Gin@nat@height>\textheight\textheight\else\Gin@nat@height\fi}
\makeatother
% Scale images if necessary, so that they will not overflow the page
% margins by default, and it is still possible to overwrite the defaults
% using explicit options in \includegraphics[width, height, ...]{}
\setkeys{Gin}{width=\maxwidth,height=\maxheight,keepaspectratio}
% Set default figure placement to htbp
\makeatletter
\def\fps@figure{htbp}
\makeatother

\usepackage{amsmath, xparse}
\usepackage{fancyvrb, fvextra}
\usepackage{unicode-math}
\usepackage{svg}
\usepackage{multicol}
\usepackage{listings}
\usepackage{systeme}
\usepackage{bm}
\usepackage{xifthen}
\DefineVerbatimEnvironment{Highlighting}{Verbatim}{breaklines,commandchars=\\\{\}}
\lstset{basicstyle=\ttfamily\footnotesize,breaklines=true}
\newcommand\rowop[1]{\scriptstyle\smash{\xrightarrow[\vphantom{#1}]{\mkern-4mu#1\mkern-4mu}}}
\DeclareDocumentCommand\converttorows%
{>{\SplitList{,}}m}%
{\ProcessList{#1}{\converttorow}}
\NewDocumentCommand{\converttorow}{m}
{\ifthenelse{\isempty{#1}}{}{\rowop{#1}}\\}

\DeclareDocumentCommand \rowops{m}
{\;
\begin{matrix}
\converttorows {#1}
\end{matrix}
\; }
\KOMAoption{captions}{tableheading}
\makeatletter
\makeatother
\makeatletter
\makeatother
\makeatletter
\@ifpackageloaded{caption}{}{\usepackage{caption}}
\AtBeginDocument{%
\ifdefined\contentsname
  \renewcommand*\contentsname{Table of contents}
\else
  \newcommand\contentsname{Table of contents}
\fi
\ifdefined\listfigurename
  \renewcommand*\listfigurename{List of Figures}
\else
  \newcommand\listfigurename{List of Figures}
\fi
\ifdefined\listtablename
  \renewcommand*\listtablename{List of Tables}
\else
  \newcommand\listtablename{List of Tables}
\fi
\ifdefined\figurename
  \renewcommand*\figurename{Figure}
\else
  \newcommand\figurename{Figure}
\fi
\ifdefined\tablename
  \renewcommand*\tablename{Table}
\else
  \newcommand\tablename{Table}
\fi
}
\@ifpackageloaded{float}{}{\usepackage{float}}
\floatstyle{ruled}
\@ifundefined{c@chapter}{\newfloat{codelisting}{h}{lop}}{\newfloat{codelisting}{h}{lop}[chapter]}
\floatname{codelisting}{Listing}
\newcommand*\listoflistings{\listof{codelisting}{List of Listings}}
\makeatother
\makeatletter
\@ifpackageloaded{caption}{}{\usepackage{caption}}
\@ifpackageloaded{subcaption}{}{\usepackage{subcaption}}
\makeatother
\makeatletter
\@ifpackageloaded{tcolorbox}{}{\usepackage[skins,breakable]{tcolorbox}}
\makeatother
\makeatletter
\@ifundefined{shadecolor}{\definecolor{shadecolor}{rgb}{.97, .97, .97}}
\makeatother
\makeatletter
\makeatother
\makeatletter
\makeatother
\ifLuaTeX
  \usepackage{selnolig}  % disable illegal ligatures
\fi
\IfFileExists{bookmark.sty}{\usepackage{bookmark}}{\usepackage{hyperref}}
\IfFileExists{xurl.sty}{\usepackage{xurl}}{} % add URL line breaks if available
\urlstyle{same} % disable monospaced font for URLs
\hypersetup{
  colorlinks=true,
  linkcolor={blue},
  filecolor={Maroon},
  citecolor={Blue},
  urlcolor={Blue},
  pdfcreator={LaTeX via pandoc}}

\author{}
\date{}

\begin{document}
\input{./title.tex}
\newpage

\ifdefined\Shaded\renewenvironment{Shaded}{\begin{tcolorbox}[borderline west={3pt}{0pt}{shadecolor}, boxrule=0pt, sharp corners, enhanced, interior hidden, breakable, frame hidden]}{\end{tcolorbox}}\fi

\renewcommand*\contentsname{Table of Contents}
{
\hypersetup{linkcolor=}
\setcounter{tocdepth}{4}
\tableofcontents
}
\newpage{}

\hypertarget{does-the-following-set-of-column-vectors-form-a-basis-for-mathbfr3}{%
\section{\texorpdfstring{Does the following set of column vectors form a
basis for
\(\mathbf{R^3}\)?}{Does the following set of column vectors form a basis for \textbackslash mathbf\{R\^{}3\}?}}\label{does-the-following-set-of-column-vectors-form-a-basis-for-mathbfr3}}

\(\symbf{\left\{ \ \ \begin{bmatrix}1 \\ 0 \\ 2\end{bmatrix}, \begin{bmatrix}3 \\ 2 \\ -4\end{bmatrix}, \begin{bmatrix}-3 \\ -5 \\ 1\end{bmatrix} \ \ \right\}}\)

Point out \textbf{\emph{each}} of the requirements for whether a set is
a basis and how these three vectors meet or do not meet each of them.

If it succeeds to form a basis, tell me how it fulfills each
requirement. If it fails, tell me how it fails and succeeds for each
requirement.

\hypertarget{requirement-1-the-set-of-vectors-are-linearly-independent}{%
\subsection{Requirement 1: The set of vectors are linearly
independent}\label{requirement-1-the-set-of-vectors-are-linearly-independent}}

Let's put the vectors into a matrix and check the determinant:
\begin{align*}
&\det\left(\begin{bmatrix}1 & 3 & -3 \\ 0 & 2 & -5 \\ 2 & -4 & 1 \end{bmatrix}\right) \\
&= \det\left(\begin{bmatrix}2 & -5 \\ -4 & 1\end{bmatrix}\right)-3\det\left(\begin{bmatrix}0 & -5 \\ 2 & 1\end{bmatrix}\right) + (-3)\det\left(\begin{bmatrix}0 & 2 \\ 2 & -4\end{bmatrix}\right) \\
&= (2\cdot 20)-3(0\cdot -10)-3(0\cdot -4) \\
&= 40-3(0)-3(0) \\
&= 40 
\end{align*} Since \(40 \ne 0\), the vectors are \textbf{linearly
independent}.

\hypertarget{requirement-2-the-set-of-vectors-span-r3}{%
\subsection{\texorpdfstring{Requirement 2: The set of vectors span
\(R^3\)}{Requirement 2: The set of vectors span R\^{}3}}\label{requirement-2-the-set-of-vectors-span-r3}}

Let's check the rank of the previous matrix: \begin{align*}
\mathrm{rref}\left(\begin{bmatrix}1 & 3 & -3 \\ 0 & 2 & -5 \\ 2 & -4 & 1 \end{bmatrix}\right) &= \begin{bmatrix}1 & 0 & 0 \\ 0 & 1 & 0 \\ 0 & 0 & 1 \end{bmatrix}
\end{align*} Since the rank is 3, the vectors \textbf{span
\(\symbf{R^3}\)}.

Therefore, the set of column vectors \emph{do} form a \textbf{basis for
\(\symbf{R^3}\)}.

\newpage{}

\hypertarget{symbfa-is-a-non-zero-symbf11-times-7-matrix}{%
\section{\texorpdfstring{\(\symbf{A}\) is a non-zero
\(\symbf{11 \times 7}\)
matrix}{\textbackslash symbf\{A\} is a non-zero \textbackslash symbf\{11 \textbackslash times 7\} matrix}}\label{symbfa-is-a-non-zero-symbf11-times-7-matrix}}

What is the largest possible \textbf{\emph{rank}} of \(A\)? What is the
smallest? \textbf{Why?}

The \textbf{largest possible rank} of \(A\) is
\(\mathrm{min}⁡(11,7) = \symbf{7}\) because the minimum of the number of
rows and columns determines the maximum number of linearly independent
rows. The \textbf{smallest possible rank} of \(A\) is \textbf{1}, which
occurs when all the rows/columns of \(A\) are multiples of each other.

What is the largest possible \textbf{\emph{nullity}} of \(A\)?
\textbf{Why?}

The largest possible nullity of \(A\) is the maximum number of linearly
independent columns that can be added to \(A\) before it becomes
singular. This gives us the following relationship:
\(\text{maximum nullity}+\text{minimum rank}=\text{number of rows}\).
Because the minimum rank is 1, \textbf{the maximum nullity is 6} because
\(1+6=7\).

\newpage{}

Make me a particular example of the above matrix, \(\symbf{A}\), that
has a rank of \textbf{4}.

\(\begin{bmatrix} 1 & 0 & 0 & 0 & 0 & 0 & 0\\ 0 & 1 & 0 & 0 & 0 & 0 & 0 \\ 0 & 0 & 1 & 0 & 0 & 0 & 0 \\ 0 & 0 & 0 & 1 & 0 & 0 & 0 \\ 0 & 0 & 0 & 0 & 0 & 0 & 0 \\ 0 & 0 & 0 & 0 & 0 & 0 & 0 \\ 0 & 0 & 0 & 0 & 0 & 0 & 0 \\ 0 & 0 & 0 & 0 & 0 & 0 & 0 \\ 0 & 0 & 0 & 0 & 0 & 0 & 0 \\ 0 & 0 & 0 & 0 & 0 & 0 & 0 \\ 0 & 0 & 0 & 0 & 0 & 0 & 0 \end{bmatrix}\)

What is the nullity of your example?

The \textbf{nullity} is \textbf{3}.

\newpage{}

\hypertarget{symbfa-beginbmatrix2-2-1-4--1-6-4-4-6-0-5--8--4--4--10-8--12-28-0-0--12-24--21-60endbmatrix}{%
\section{\texorpdfstring{\(\symbf{A = \begin{bmatrix}2 & 2 & 1 & 4 & -1 & 6 \\ 4 & 4 & 6 & 0 & 5 & -8 \\ -4 & -4 & -10 & 8 & -12 & 28 \\ 0 & 0 & -12 & 24 & -21 & 60\end{bmatrix}}\)}{\textbackslash symbf\{A = \textbackslash begin\{bmatrix\}2 \& 2 \& 1 \& 4 \& -1 \& 6 \textbackslash\textbackslash{} 4 \& 4 \& 6 \& 0 \& 5 \& -8 \textbackslash\textbackslash{} -4 \& -4 \& -10 \& 8 \& -12 \& 28 \textbackslash\textbackslash{} 0 \& 0 \& -12 \& 24 \& -21 \& 60\textbackslash end\{bmatrix\}\}}}\label{symbfa-beginbmatrix2-2-1-4--1-6-4-4-6-0-5--8--4--4--10-8--12-28-0-0--12-24--21-60endbmatrix}}

\begin{align*}
\mathrm{rref}(A) = \begin{bmatrix}2 & 2 & 1 & 4 & -1 & 6 \\ 4 & 4 & 6 & 0 & 5 & -8 \\ -4 & -4 & -10 & 8 & -12 & 28 \\ 0 & 0 & -12 & 24 & -21 & 60 \end{bmatrix} \rowops{\frac{1}{2}R_1,,,} \begin{bmatrix} 1 & 1 & \frac{1}{2} & 2 & -\frac{1}{2} & 3 \\ 4 & 4 & 6 & 0 & 5 & -8 \\ -4 & -4 & -10 & 8 & -12 & 28 \\ 0 & 0 & -12 & 24 & -21 & 60 \end{bmatrix} \rowops{,R_2-4R_1,R_3+4R_1,} \\
\begin{bmatrix} 1 & 1 & \frac{1}{2} & 2 & -\frac{1}{2} & 3 \\ 0 & 0 & 4 & -8 & 7 & -20 \\ 0 & 0 & -8 & 16 & -14 & 40 \\ 0 & 0 & -12 & 24 & -21 & 60 \end{bmatrix} \rowops{,,R_3+2R_2,} \begin{bmatrix} 1 & 1 & \frac{1}{2} & 2 & -\frac{1}{2} & 3 \\ 0 & 0 & 4 & -8 & 7 & -20 \\ 0 & 0 & 0 & 0 & 0 & 0 \\ 0 & 0 & -12 & 24 & -21 & 60 \end{bmatrix} \rowops{,,,R_4+3R_2} \\
\begin{bmatrix} 1 & 1 & \frac{1}{2} & 2 & -\frac{1}{2} & 3 \\ 0 & 0 & 4 & -8 & 7 & -20 \\ 0 & 0 & 0 & 0 & 0  & 0 \\ 0 & 0 & 0 & 0 & 0 & 0 \end{bmatrix} \rowops{,\frac{1}{4}R_2,,}\begin{bmatrix}1 & 1 & \frac{1}{2} & 2 & -\frac{1}{2} & 3 \\ 0 & 0 & 1 & -2 & \frac{7}{4} & -5 \\ 0 & 0 & 0 & 0 & 0 & 0 \\ 0 & 0 & 0 & 0 & 0 & 0 \end{bmatrix} \rowops{R_1-\frac{1}{2}R_2,,,} \begin{bmatrix} 1 & 1 & 0 & 3 & -\frac{11}{8} & \frac{11}{2} \\ 0 & 0 & 1 & -2 & \frac{7}{4} & -5 \\ 0 & 0 & 0 & 0 & 0 & 0 \\ 0 & 0 & 0 & 0 & 0 & 0 \end{bmatrix}
\end{align*}

Find \ul{\textbf{and justify}} each of the following:

\begin{enumerate}
\def\labelenumi{\alph{enumi}.}
\tightlist
\item
  Set of vectors that form a basis for the column space of \(A\)
\end{enumerate}

We know that the column space of \(A\) is equivalent to:

\(\mathrm{span}\left(\begin{bmatrix}1 \\ 0 \\ 0 \\ 0\end{bmatrix},\begin{bmatrix}1 \\ 0 \\ 0 \\ 0\end{bmatrix},\begin{bmatrix}0\\1\\0\\0\end{bmatrix},\begin{bmatrix}3\\-2\\0\\0\end{bmatrix},\begin{bmatrix}-\frac{11}{8}\\ \frac{7}{4}\\0\\0\end{bmatrix},\begin{bmatrix}\frac{11}{2}\\-5\\0\\0\end{bmatrix}\right)\)

But we need to simplify that span some more. So, we just need to
identify which vectors in the span are linearly independent. We can do
this by looking at the rref of \(A\) and identifying which columns have
pivots. The columns with pivots are the linearly independent columns,
and the columns without pivots are linearly dependent. Therefore, the
column space of \(A\) is:

\(\left\{\begin{bmatrix}1 \\ 0 \\ 0 \\ 0\end{bmatrix},\begin{bmatrix}0 \\ 1 \\ 0 \\ 0\end{bmatrix}\right\} \Rightarrow \left\{\begin{bmatrix}2 \\ 4 \\ -4 \\ 0\end{bmatrix},\begin{bmatrix}1 \\ 6 \\ -10 \\ -12 \end{bmatrix}\right\}\)

\newpage{}

\begin{enumerate}
\def\labelenumi{\alph{enumi}.}
\setcounter{enumi}{1}
\tightlist
\item
  Set of vectors that form a basis for the row space of \(A\)
\end{enumerate}

We know that the row space of \(A\) is equivalent to:

\(\mathrm{span}\left(\begin{bmatrix}1 \\ 1 \\ 0 \\ 3 \\ -\frac{11}{8} \\ \frac{11}{2}\end{bmatrix},\begin{bmatrix}0 \\ 0 \\ 1 \\ -2 \\ \frac{7}{4} \\ -5\end{bmatrix},\begin{bmatrix}0 \\ 0 \\ 0 \\ 0 \\ 0 \\ 0\end{bmatrix} \right)\)

But we need to simplify that span some more. So, we just need to
identify which vectors in the span are linearly independent. We can do
this by looking at the rref of \(A\) and identifying which rows are
non-zero. The non-zero rows are the linearly independent rows, and the
zero rows are linearly dependent. Therefore, the row space of \(A\) is:

\(\left\{\begin{bmatrix}1 \\ 1 \\ 0 \\ 3 \\ -\frac{11}{8} \\ \frac{11}{2}\end{bmatrix},\begin{bmatrix}0 \\ 0 \\ 1 \\ -2 \\ \frac{7}{4} \\ -5\end{bmatrix}\right\}\)

\begin{enumerate}
\def\labelenumi{\alph{enumi}.}
\setcounter{enumi}{2}
\tightlist
\item
  Set of vectors that form a basis for the null space of \(A\)
\end{enumerate}

We can represent \(A\) as a system of equations:

\systeme*{x_1 = -x_2-3x_4+\frac{11}{8}x_5-\frac{11}{2}x_6, x_2 = x_2, x_3 = 2x_4-\frac{7}{4}x_5+5x_6, x_4=x_4, x_5=x_5, x_6=x_6}

\(\Rightarrow \vec{x} = \begin{bmatrix}x_1 \\ x_2 \\ x_3 \\ x_4 \\ x_5 \\ x_6\end{bmatrix}=\begin{bmatrix}-x_2-3x_4+\frac{11}{8}x_5-\frac{11}{2}x_6 \\ x_2 \\ 2x_4-\frac{7}{4}x_5+5x_6 \\ x_4 \\ x_5 \\ x_6\end{bmatrix}=x_2\begin{bmatrix}-1 \\ 1 \\ 0 \\ 0 \\ 0 \\ 0\end{bmatrix}+x_4\begin{bmatrix}-3 \\ 0 \\ 2 \\ 1 \\ 0 \\ 0\end{bmatrix}+x_5\begin{bmatrix}\frac{11}{8}\\ 0 \\ -\frac{7}{4} \\ 0 \\ 1 \\ 0\end{bmatrix}+x_6\begin{bmatrix}-\frac{11}{2} \\ 0 \\ 5 \\ 0 \\ 0 \\ 1\end{bmatrix}\)

\newpage{}

Therefore, the basis of the null space can be represented with the
following set:

\(\left\{\begin{bmatrix}-1 \\ 1 \\ 0 \\ 0 \\ 0 \\ 0\end{bmatrix}, \begin{bmatrix}-3 \\ 0 \\ 2 \\ 1 \\ 0 \\ 0\end{bmatrix}, \begin{bmatrix}\frac{11}{8}\\ 0 \\ -\frac{7}{4} \\ 0 \\ 1 \\ 0\end{bmatrix}, \begin{bmatrix}-\frac{11}{2} \\ 0 \\ 5 \\ 0 \\ 0 \\ 1\end{bmatrix}\right\}\)

\begin{enumerate}
\def\labelenumi{\alph{enumi}.}
\setcounter{enumi}{3}
\tightlist
\item
  Rank of \(A\)
\end{enumerate}

The rank of \(A\) is just the \textbf{dimension} of the column space
basis of \(A\). There are two elements in the basis set for the column
space of \(A\), so the rank of \(A\) is \textbf{2}.

\begin{enumerate}
\def\labelenumi{\alph{enumi}.}
\setcounter{enumi}{4}
\tightlist
\item
  Nullity of \(A\)
\end{enumerate}

The nullity of \(A\) is just the \textbf{dimension} of the null space
basis of \(A\). There are four elements in the basis set for the null
space of \(A\), so the nullity of \(A\) is \textbf{4}.

\newpage{}

\hypertarget{prove-the-following}{%
\section{Prove the following:}\label{prove-the-following}}

\textbf{If a matrix \(\symbf{A}\) is not square, then either the row
vectors or the column vectors of \(A\) are linearly dependent.}

I'm going to take an unconventional approach to this proof. Consider a
matrix \(A\) with dimensions \(m \times n\) where \(m > n\). Let
\(\symbf{r_1, r_2, ..., r_m}\) represent the row vectors of \(A\). If
any of the row vectors are the zero vector or equal, they are linearly
dependent by default.

In that case, let's assume none of the row vectors are zero vectors or
equivalent to one another. Since \(m \ne n\), at least two row vectors
must have more components than the possible columns of the matrix by the
Pigeonhole Principle. This implies that there exist constants
\(c_1, c_2, ... c_m\) not all zero, such that:

\(c_1\symbf{r_1}+c_2\symbf{r_2}+ ... + c_m\symbf{r_m} = \symbf{0}\)

This is a non-trivial linear combination of the row vectors that equals
the zero vector, making the row vectors \textbf{linearly dependent.}

But that's only for \(m > n\). What about \(m < n\)?

Well, similar to the approach for the row vectors, if any of the column
vectors is a zero vector, they are linearly dependent. Otherwise, if any
two column vectors are equal, they are linearly dependent because one
can be expressed as a multiple of the other. By the Pigeonhole
Principle, at least two column vectors must have more components than
the number of rows. This implies that there exist constants
\(d_1, d_2, ..., d_n\) not all zero, such that:

\(d_1\symbf{c_1}+d_2\symbf{c_2}+ ... +d_n\symbf{c_n} = \symbf{0}\)

Yet again, this is a non-trivial linear combination of the column
vectors that equals the zero vector, making the column vectors linearly
dependent.

Therefore, if \(A\) is not square, at least one set of vectors
(depending on the values of \(m\) and \(n\)) must be linearly dependent.

\newpage{}

\hypertarget{can-you-go-backwards}{%
\section{Can you go backwards?}\label{can-you-go-backwards}}

I have a matrix that has a \ul{\textbf{row space}} given by all vectors
of the form:

\(\langle -4t + 5w - 3s, 2w + s, 3t + 4w \rangle\)

Provide for me a matrix that has this row space.

We can decomponentize the vectors to get:

\(\begin{bmatrix}-4 \\ 0 \\ 3\end{bmatrix}t + \begin{bmatrix}5 \\ 2 \\ 4\end{bmatrix}w + \begin{bmatrix}-3 \\ 1 \\ 0\end{bmatrix}s\)

But wait a minute! That literally looks like a span! We could just say
this:

\(\begin{bmatrix}-4 \\ 0 \\ 3\end{bmatrix}t + \begin{bmatrix}5 \\ 2 \\ 4\end{bmatrix}w + \begin{bmatrix}-3 \\ 1 \\ 0\end{bmatrix}s \Rightarrow \mathrm{span}\left(\begin{bmatrix}-4 \\ 0 \\ 3\end{bmatrix}, \begin{bmatrix}5 \\ 2 \\ 4\end{bmatrix}, \begin{bmatrix}-3 \\ 1 \\ 0\end{bmatrix}\right)\)

Since we know the vectors span the row space, we can just put them into
a matrix:

\(\begin{bmatrix}-4 & 0 & 3 \\ 5 & 2 & 4 \\ -3 & 1 & 0\end{bmatrix}\)

\newpage{}

\newpage{}

\hypertarget{connect-the-following-equivalent-statements-together-with-an-explanation-of-how-one-statement-implies-the-other.}{%
\section{Connect the following equivalent statements together with an
explanation of how one statement implies the
other.}\label{connect-the-following-equivalent-statements-together-with-an-explanation-of-how-one-statement-implies-the-other.}}

\(\det{(A)} \neq 0 \rightarrow A \text{ has nullity 0} \rightarrow \text{The columns of } A \text{ are linearly independent} \rightarrow\)

\(\text{The column vectors of } A \text{ span } R^n \rightarrow \symbf{A}x = b \text{ has exactly one solution for every } n \times 1 \text{ matrix } b \rightarrow\)

\hypertarget{symbfdeta-neq-0}{%
\subsection{\texorpdfstring{\(\symbf{\det{(A)} \neq 0}\)}{\textbackslash symbf\{\textbackslash det\{(A)\} \textbackslash neq 0\}}}\label{symbfdeta-neq-0}}

Because \(\det{(A)} \neq 0\), \(A\) is invertible and non-singular, so
the nullity of \(A\) must be 0.

\hypertarget{symbfa-has-nullity-0}{%
\subsection{\texorpdfstring{\(\symbf{A}\) has nullity
0}{\textbackslash symbf\{A\} has nullity 0}}\label{symbfa-has-nullity-0}}

For any matrix with a nullity of 0, the null space, representing
solutions to \(\symbf{A}x = \symbf{0}\), must only contain the trivial
solution \(x = 0\), implying linear independence in the column space.

\hypertarget{the-columns-of-symbfa-are-linearly-independent}{%
\subsection{\texorpdfstring{The columns of \(\symbf{A}\) are linearly
independent}{The columns of \textbackslash symbf\{A\} are linearly independent}}\label{the-columns-of-symbfa-are-linearly-independent}}

Linearly independent columns ensure that the column vectors span the
entire space \(R^n\). If there were linear dependence, it would mean
redundancy in the columns and, consequently, the inability to span the
entire column space.

\hypertarget{the-column-vectors-of-symbfa-span-symbfrn}{%
\subsection{\texorpdfstring{The column vectors of \(\symbf{A}\) span
\(\symbf{R^n}\)}{The column vectors of \textbackslash symbf\{A\} span \textbackslash symbf\{R\^{}n\}}}\label{the-column-vectors-of-symbfa-span-symbfrn}}

Because the column vectors of \(A\) span \(R^n\), any vector in \(R^n\)
can be expressed as a unique linear combination of these columns.
Therefore, the system \(\symbf{A}x = b\) has a unique solution for every
\(n \times 1\) matrix \(b\).

\hypertarget{symbfax-b-has-exactly-one-solution-for-every-symbfn-times-1-matrix-symbfb}{%
\subsection{\texorpdfstring{\(\symbf{Ax = b}\) has exactly one solution
for every \(\symbf{n \times 1}\) matrix
\(\symbf{b}\)}{\textbackslash symbf\{Ax = b\} has exactly one solution for every \textbackslash symbf\{n \textbackslash times 1\} matrix \textbackslash symbf\{b\}}}\label{symbfax-b-has-exactly-one-solution-for-every-symbfn-times-1-matrix-symbfb}}

If the determinant of \(A\) is zero, it implies that the matrix is
singular, and the system of equations may have either no solution or
infinitely many solutions, but it won't have a unique solution for every
\(b\). Therefore, for the system \(\symbf{A}x = b\) to have a unique
solution for every \(b\), the determinant of matrix \(A\) must be
nonzero.



\end{document}
